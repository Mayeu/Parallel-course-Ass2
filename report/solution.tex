\chapter{Solution method}

\section{main()}

Take the size of the array and the number of step (number of thread is $2^{step}$ at most):

\begin{verbatim}
check array size
check number of step

allocate the array
fill the array with random data

struct qsort_arg = arg
arg.step = 0
arg.begin = 0
arg.end = size-1

spawn the first thread (quicksort(arg) )

join the thread

check the result

deallocate data
\end{verbatim}

\section{quicksort()}

This function take a qsort\_arg structure in argument. This structure hold:
\begin{itemize}
  \item the beginning of the work area
  \item the end of the work area
  \item the actual working step
\end{itemize}

\begin{verbatim}
if arg.step >= max_step
    serial_quicksort (array, arg.begin, arg.end)

else
    index = permut(array, arg.begin, arg.end)
    if arg.end > index + 1
        struct qsort\_arg arg2
        arg2.begin = index + 1
        arg2.end = arg.end
        arg2.step = arg.step + 1
        spawn new thread (quicksort(arg2)

    if arg.begin < index - 1
        arg.step++
        arg.end = index - 1
        quicksort(arg)

    if a thread was spawned
        join this thread
\end{verbatim}

\section{permut()}

This function permut elements of an array around a choosen pivot element. It take an array, the begin and the end for the work.

\begin{verbatim}
choose pivot

while begin != end
    if array[begin] >= array[end] 
        swap(array, begin,end)
        pivot =  begin+end - pivot
	 
    if pivot == begin
        end --
    else
        begin++

return begin
\end{verbatim}

\section{serial\_quicksort()}

This function sort an array. It take the array and the beginning and the end for the work.

\begin{verbatim}
index = permut (array, begin, end)

if begin < index - 1
    serial_quicksort (array, begin, index - 1)
if end > index + 1
    serial_quicksort (array, index + 1, end)
\end{verbatim}
